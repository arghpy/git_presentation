\documentclass{beamer}

\usetheme{Frankfurt}

\title{Git and GitHub}
\subtitle{A short introduction}

\author{Stefan Bogdan Isac \and Andrei Suba \and Aba Farkas}

\date[WS 2024]
{Internal Workshop, November 2024}
% Multiple logos
\titlegraphic{
    \includegraphics[height=1cm]{./images/ibm/ibm.png}
    \hspace{1cm}
    \includegraphics[height=1cm]{./images/github/github.png}
}

\begin{document}
\frame{\titlepage}

\begin{frame}
\frametitle{Table of Contents}
\tableofcontents
\end{frame}

\section{Initial Setup}
\begin{frame}
\frametitle{Initial Setup}

In this initial setup we will create an environment in which
we can use the \textit{git} command, may it be on
Windows, macOS or Linux.
\end{frame}

\subsection{Windows}
\begin{frame}
\frametitle{Windows}

Take into account the following considerations:
\begin{itemize}
  \item First consideration
  \item Second consideration
  \item Third consideration
\end{itemize}
  
\end{frame}

\subsection{macOS}
\begin{frame}
\frametitle{macOS}

  Open a terminal by accessing the \textbf{Applications} window.
  Then simply run the following command:
  \begin{examples}
    brew install git
  \end{examples}
  
\end{frame}

\subsection{Linux}
\begin{frame}
\frametitle{Linux}

By using your Distro's package manager, simply install git.
  \begin{alertblock}{Important}
    Make sure to run the command as an administrator.\\
    Use \alert{sudo}.
  \end{alertblock}
  
\end{frame}

\end{document}
